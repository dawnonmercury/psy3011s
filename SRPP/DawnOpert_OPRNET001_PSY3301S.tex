\documentclass[stu,a4paper,12pt,donotrepeattitle]{apa7}
\usepackage{csquotes}
\usepackage[english]{babel}
\usepackage[
backend=biber,
style=apa,
]{biblatex}
\addbibresource{/home/dawn/.bibliography/psy3011s.bib}

\title{SRPP Catch-Up Task: Second Semester\\~\\
1 SRPP Point}
\author{Dawn Opert}
\affiliation{OPRNET001}
\course{PSY3011S: Clinical Psychology II}
\professor{Katya Kee-Tui}
\duedate{15 October 2022}

\begin{document}
\maketitle
\section{Plagiarism Declaration}
\begin{enumerate}
    \item I know that plagiarism is wrong. Plagiarism means to present
        substantial portions or elements of another’s work, ideas or data
        as my own, even if the original author is cited occasionally.
    \item I have used the American Psychological Association formatting for
        citation and referencing. Each significant contribution to, and
        quotation in, this essay/project/report from the work or works of other
        people has been attributed, cited and referenced.
    \item This essay/project/report is my own work.
    \item \textbf{I have not allowed, and will not allow anyone to copy my work
        with the intention of passing it off as their own work.}
\end{enumerate}

\textbf{Signed:} \underline{Dawn Opert}\\
\textbf{Date:} \underline{15 October 2022}
\newpage
\section{Introduction}
Depression is one of the most prevalent mental health issues across the globe,
and is responsible for an enormous amount of suffering caused by disability
(\cite{makh21b}; \cite{makh21}; \cite{makhdeb16}; \cite{malletal18};
\cite{mason19}; \cite{roussetal21}). Over the last decade, there has also been
robust evidence that both the rate of prevalence and the severity of depression
has been rising rapidly (\cite{makh21}; \cite{roussetal21}). Furthermore,
university students represent a group that are uniquely vulnerable to
depression and other mood disorders, and often do not have the emotional,
financial, and social resources to deal it (\cite{makh21}; \cite{malletal18};
\cite{mason19}; \cite{roussetal21}).\\
Despite depression among university students being a serious concern in South
Africa, much of the literature has been written by and about high income
countries (\cite{makh21b}; \cite{makh21}; \cite{makhdeb16}; \cite{roussetal21}).
As will be discussed, there has been a recent number of papers published about
the South African context, but such literature has generally been few and far
between.\\
This paper will review the literature on depression in university students in
South Africa. First, the details of the literature itself will be discussed. Next,
there will be discussion of two themes in the literature: Causes of student
depression, and gender and depression.
Finally, the paper will conclude with a suggestion of directions for future
research.
\section{Literature Review}
Six papers in total were found to be acceptable. Ten papers were found on the
UCT Libraries Primo database, by searching for "Depression in South African
university students", "University students with depression in South Africa",
and by searching the reference lists of papers that were already found. Of
those ten papers, four did not refer to South Africa, rather talking about
lower and middle-income countries generally, or referring to other southern
African countries such as Botswana. Of these six, Makhubela (\citeyear{makh21})
is not a research paper, but rather a background report. There are five
(\cite{makh21b}; \cite{makhdeb16}; \cite{malletal18}; \cite{mason19};
\cite{roussetal21}) research papers. All five were conducted in South African
universities. There were vanishingly few useful papers, and four of the five
research papers were published in the last four years.\\
One of the main reasons that university students are particularly vulnerable to
depression is that they face many new stressors, such as strict academic
requirements, intense financial pressures, and often new social pressures such
as gender, sexuality, and identity exploration (\cite{malletal18};
\cite{roussetal21}). While some of these can be ultimately positive, such as
exploration of identity, most students are young (between 18-24) and often do
not have the emotional experience to navigate many of these changes in their
lives, nor are they given the support that they need to develop the necessary
emotional skills (\cite{mason19}; \cite{roussetal21}). In South Africa this
becomes precarious, as severe inequalities mean that most students will be
coming from families that do not have stable financial situations, many will be
the first university student in their families, and many will have experienced
trauma (\cite{malletal18}; \cite{roussetal21}). These factors add up to create a
particularly vulnerable group who often are required to adapt fast or simply
drop out (\cite{roussetal21}).\\
With regards to gender, there is a literature consensus that men and women
experience depression at different rates of prevalence, with women tending to
experience higher rates of depression (\cite{makh21b}; \cite{makhdeb16}).
Makhubela (\citeyear{makh21b}) finds that there does not seem to be any
qualitative difference between the depression of women compared to the
depression of men, with the only real difference being in prevalence rates. This
lends further support to the notion that the cause of the higher prevalence rate
is external, often living within a patriarchal system, and socialization which
encourages negative self-evaluation and ruminative coping (\cite{makh21b};
\cite{makhdeb16}). It is, however, disappointing to see that gender was only
discussed in these papers as far as a gender binary, and in regards only to
prevalence rates rather than research of experiences. Mall et al.
(\citeyear{malletal18}) specifically note that a gender crisis is one of the
many factors that increase a student's risk of depression, but there is no
discussion of what this looks like or why this increases risk. In fact, none of
the six papers reviewed in this paper discuss any experiences at all.
\section{Conclusion}
There is frightfully little written on depression as it manifests in university
students in South Africa. Makhubela (\citeyear{makh21b}; \cite{makh21}) and
Rousseau et al. (\citeyear{roussetal21}) were both published in the last two
years, so there does seem to be a movement towards more research on this very
unique population. However, what is written seems to be almost scattered, and
there are a number of gaping blindspots. Firstly, future research should
incorporate qualitative studies of students' experiences, both with depression
and with university. By speaking directly with students, and by allowing students
to be equal partners in studying the risks that the university environment
subjects them to, we stand to learn more about the ways that depression
actually and tangibly affect students. This allows universities to develop
more responsive programs to help support students, and minimise that risk.
\newpage
\printbibliography
\end{document}
