
\documentclass[stu,a4paper,12pt,donotrepeattitle]{apa7}
\usepackage{csquotes}
\usepackage[english]{babel}
\usepackage[
backend=biber,
style=apa,
]{biblatex}
\addbibresource{/home/dawn/.bibliography/psy3011s.bib}
\addbibresource{/home/dawn/.bibliography/psy2014s.bib}

\title{Tutorial Assignment 2}
\author{Dawn Opert}
\affiliation{OPRNET001}
\course{PSY3011S: Clinical Psychology II}
\professor{Katya Kee-Tui}
\duedate{26 September 2022}

\begin{document}
\maketitle
\section{Plagiarism Declaration}
\noindent{\fbox{
        \parbox{\textwidth}{
            \center{\textbf{PLAGIARISM}}

            \flushleft
            This means that you present substantial
            portions or elements of another’s work, ideas or data as your own,
            even if the original author is cited occasionally. A signed
            photocopy or other copy of the Declaration below must accompany
            every piece of work that you hand in.
        }
}}\\[1in]
\begin{enumerate}
    \item I know that plagiarism is wrong. Plagiarism is to use the work of
        another and pass it off as my own.
    \item I have used the American Psychological Association formatting for
        citation and referencing. Each significant contribution to, and
        quotation in, this essay/project/report from the work or works of other
        people has been attributed, cited and referenced.
    \item This essay/project/report is my own work.
    \item \textbf{I have not allowed, and will not allow anyone to copy my work
        with the intention of passing it off as their own work.}
\end{enumerate}

\textbf{Signed:} \underline{Dawn Opert}\\
\textbf{Date:} \underline{25 September 2022}
\newpage
\section{Introduction}
The addition of Disruptive Mood Dysregulation Disorder (DMDD) to the DSM-5 has
been the subject of intense debate and controversy in the years following.
Whether the disorder should be a part of the DSM, whether it solves the problem
it was introduced to tackle, and what alternative options we have are the
subject of intense debates that are directly relevant to the care afforded to
children and adolescents in need. This essay will be looking at the context
surrounding DMDD, and examining the various arguments for and against its
inclusion.\\
First, this essay will lay out the background for the inclusion of DMDD and the
initial rationale behind it. Afterwards, this essay will present and discuss
why DMDD should remain in the DSM and why it should not. Finally, the essay
will summarise the literature, and I will explain why I think that DMDD fails
to adequately solve the problem it was created to deal with, and outline some
potential alternatives.\\
DMDD is a mood disorder that is characterised by severe temper outbursts and
persistent irritability, most of the time, across multiple contexts
(\cite{baweja16}; \cite{franc13}; \cite{lochman15}). When it was initially
included in the DSM-5, there was a growing recognition of a pandemic of
over-diagnosis of Bipolar Disorder among children who displayed tempers
(\cite{baweja16}; \cite{lochman15}). This tended to weigh these children down
with the implications of a Bipolar diagnosis, and unnecessary and distressing
treatments (\cite{franc13}; \cite{lochman15}). DMDD was meant to fix this.
\section{Advantages and Disadvanteges}
\section{Conclusion}
\newpage
\printbibliography
\end{document}

