
\documentclass[stu,a4paper,12pt,donotrepeattitle]{apa7}
\usepackage{csquotes}
\usepackage[english]{babel}
\usepackage[
backend=biber,
style=apa,
]{biblatex}
\addbibresource{/home/dawn/.bibliography/psy3011s.bib}

\title{Tutorial Assignment 2}
\author{Dawn Opert}
\affiliation{OPRNET001}
\course{PSY3011S: Clinical Psychology II}
\professor{Katya Kee-Tui}
\duedate{26 September 2022}

\begin{document}
\maketitle
\section{Plagiarism Declaration}
\noindent{\fbox{
        \parbox{\textwidth}{
            \center{\textbf{PLAGIARISM}}

            \flushleft
            This means that you present substantial
            portions or elements of another’s work, ideas or data as your own,
            even if the original author is cited occasionally. A signed
            photocopy or other copy of the Declaration below must accompany
            every piece of work that you hand in.
        }
}}\\[1in]
\begin{enumerate}
    \item I know that plagiarism is wrong. Plagiarism is to use the work of
        another and pass it off as my own.
    \item I have used the American Psychological Association formatting for
        citation and referencing. Each significant contribution to, and
        quotation in, this essay/project/report from the work or works of other
        people has been attributed, cited and referenced.
    \item This essay/project/report is my own work.
    \item \textbf{I have not allowed, and will not allow anyone to copy my work
        with the intention of passing it off as their own work.}
\end{enumerate}

\textbf{Signed:} \underline{Dawn Opert}\\
\textbf{Date:} \underline{25 September 2022}
\newpage
\section{Introduction}
Disruptive Mood Dysregulation Disorder (DMDD) is a mood disorder that is
characterised by severe temper outbursts and
persistent irritability, most of the time, across multiple contexts
(\cite{apa13}; \cite{baweja16}; \cite{franc13}; \cite{lochman15}).
When it was initially included in the DSM-5, there was a growing recognition
of a pandemic of over-diagnosis of Bipolar Disorder among children who
displayed tempers (\cite{baweja16}; \cite{lochman15}). This tended to weigh
these children down with the implications of a Bipolar diagnosis, and
unnecessary and distressing treatments (\cite{franc13}; \cite{lochman15}).
DMDD was meant to fix this, but threatens to introduce a host of new problems
instead. Criticism of the inclusion ranges from a lack of valid studies to
support the disorder to worries that the diagnostic criteria of DMDD create
a real risk that it will be overdiagnosed (\cite{baweja16}; \cite{franc13};
\cite{lochman15})\\
As above, this essay began with the background and context of DMDD, and a
description of the disorder. Following, this essay will present and discuss why
DMDD should remain in the DSM and why it should not. Then, this essay will
summarise the literature, and I will explain why I think that DMDD fails to
adequately solve the problem it was created to deal with, and outline some
potential alternatives.\\
\section{Advantages and Disadvanteges}
As mentioned above, prior to the DSM-5 there was a broadening of the definition
for Bipolar Disorder (BD) in children in order to explain to some degree impairment
caused by irritabilit, anger, and temper outbursts (\cite{baweja16};
\cite{mahbell19}). The rising rate of diagnoses for BD in children coincided with
increased innapropriate and unsafe use of adult antipsychotics on children
(\cite{baweja16}; \cite{franc13}; \cite{lochman15}; \cite{mahbell19}). At the
same time, however, the impairment and harm that this over-diagnosis was trying
to address is real, the chronic irritability and anger, the outbursts do cause
these children harm and they should be given treatment of some kind
(\cite{baweja16}). Clearly this is an issue that needs to be resolved, there are
children that experience impairment that harms them, but they are also harmed by
the diagnosis of BD. The creation of DMDD theoretically fixes this error by
creating a new construct that describes this impairment without attaching
itself to BD and adult antipsychotics (\cite{baweja16}; \cite{franc13};
\cite{mahbell19}).\\
However, DMDD introduced a number of new problems. Firstly, the disorder was
included based on very little research of any validity, and what research did
exist was originally for a testing construct called Severe Mood Dysregulation,
which shared only some of the same criteria as DMDD (\cite{baweja16};
\cite{franc13}; \cite{lochman15}; \cite{mahbell19}). Furthermore, there is an
issue with the construct itself. The only aspect of DMDD that cannot be accounted
for in Oppositional Defiant Disorder (ODD) is irritability, to the point that if
one ignores the comorbidity restriction on DMDD, almost all children with DMDD
would simply have an ODD diagnosis (\cite{baweja16}; \cite{lochman15};
\cite{mahbell19}; \cite{mayetal16}; \cite{mayetal19}). Irritability is the only
symptom of DMDD that does not exist in ODD (\cite{apa13}). However, irritability
is a feature of many other mood and behavioural disorders, and is often a reaction
to the symptoms of another disorder (\cite{mahbell19}; \cite{mayetal16};
\cite{mayetal19}). Additionally, the criteria themselves are vague. There is no
clear definition of 'temper outbursts' and how to distinguish them from normal
expressions of a child's emotions at their development level (\cite{baweja16};
\cite{franc13}; \cite{mahbell19}). The duration and age restrictions require
that clinicians diagnose children based on the children's guardians' impressions
and their own impressions (\cite{mahbell19}). DMDD also does not have clear
treatment guidelines, and oftentimes clinicians are left to figure out how to
treat children on their own, more often than not with antipsychotics and
antidepressants (\cite{mahbell19}).
As a result, in practice there is incredibly little agreement between clinicians
on individual cases, and therefore a real threat of DMDD itself becoming massively
overdiagnosed (\cite{franc13}; \cite{lochman15}; \cite{mahbell19}). Essentially,
DMDD has created the same problem that it was originally  designed to fix.
\section{Conclusion}
While DMDD was included for a noble cause, reducing the danger an innapropriate
BD diagnosis would have on a child, its inclusion seems ill-informed. There is
an alternative route that can be taken, recommended by the task-force for the
World Health Organisation's ICD-11. Instead of an unnecessary new construct,
the ICD-11 simply added an addendum to ODD that specified whether or not the
child had irritability, the only symptom that ODD does not share with DMDD
(\cite{lochman15}; \cite{mahbell19}; \cite{mayetal16}; \cite{mayetal19}). This
was supported by robust empirical data, and achieved its goal without adding
an entirely new vector for misdiagnosis (\cite{mayetal16}; \cite{mayetal19}).
Finally, ODD already has a number of clear treatments that are useful for
children with DMDD, so there is significantly less risk of misdiagnosis leading
to severely negative outcomes (\cite{lochman15}; \cite{mayetal16}; \cite{mayetal19}).
\newpage
\printbibliography
\end{document}

