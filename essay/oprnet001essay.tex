\documentclass[stu,a4paper,12pt,donotrepeattitle]{apa7}
\usepackage{csquotes}
\usepackage[english]{babel}
\usepackage[
backend=biber,
style=apa,
]{biblatex}
\addbibresource{~/.bibliography/psy3011s.bib}
\addbibresource{~/.bibliography/psy2014s.bib}

\title{Essay}
\author{Dawn Opert}
\affiliation{OPRNET001}
\course{PSY3011S: Clinical Psychology II}
\professor{Katya Kee-Tui}
\duedate{2 September 2022}

\begin{document}
\maketitle
\section{Introduction}
Depression is a serious concern in Africa, and it is necessary to have a clear
idea about how countries across the continent are engaging with it, and what
Interventions can be effectively deployed to minimise it's impact on the mental
health burden. This essay will first discuss the prevalence of depression in
African countries. Next, this essay wil discuss what Interventions have been
tested, whether they have been effective, and where they have been effective.
The quality of the evidence for these Interventions will be analysed, before
a final reflection will be made on the state of depression Interventions and
possible future directions for research and policy makers.\\
Depression and anxiety are among the most common mental disorders globally, and
cause a disproportionate amount of the global disability based healthcare
burden, a major contributer to morbidity, and a contributer to lower standard
of life (\cite{chibandaetal15}; \cite{chibandaetal16}; \cite{douketal21};
\cite{fernaetal21}; \cite{logetal18}; \cite{lunetal14}).\\
Africa bears the brunt of the burden of common mental disorders, with
coutries across sub-Saharan Africa alone accounting for at least 19\% of the
global mental health burder (\cite{lunderal15}). In specific countries, at
least 16.5\% of adults in South Africa present with a common disorder
(\cite{lunetal14}), and at least 30\% of people attending primary healthcare
facilities in Zimbabwe present with depression and anxiety symptoms
(\cite{chibandaetal11}; \cite{chibandaetal15}).\\
Further, Africa has the highest rates of people living with HIV/AIDS, a
population that is especially vulnerable to depression as compared to a general
population, and the highest rate of people living with HIV related depression
(\cite{logetal18}; \cite{petersenetal14}). This is a serious problem because
depression is strongly associated with lower rates of antiretroviral therapy
adherance which is necessary for good management of HIV/AIDS (\cite{logetal18};
\cite{lunetal14}; \cite{petersenetal14}). On a similar front, depression
severely affects young and expecting mothers, increasing the risk of
complications during pregnancy, impairing caregivers' ability to create an
environment suitable for a child, and limiting caregivers' ability to cope with
parenthood (\cite{fernaetal21}; \cite{nyatetal16}).\\
Despite the high prevalence of depression and other common mental disorders
across Africa, there is a serious lack of access to treatment and support
(\cite{chibandaetal11}; \cite{chibandaetal15}; \cite{chibandaetal16};
\cite{douketal21}; \cite{fernaetal21}; \cite{logetal18}). This is exacerbated
by the fact that only half of African countries have a mental health policy to
speak of (\cite{lunetal14}; \cite{lunderal15}), and most governments do not have
the ability to allocate the necessary funds to implement one if they had it
(\cite{douketal21}; \cite{logetal18}). The general lack of resources and
infrastructure for providing effective treatments means that there is an urgent
need to develop interventions that work, that have lasting effects, and that can
be deployed cheaply and efficiently (\cite{douketal21}; \cite{lunetal14};
\cite{lunderal15}; \cite{Osetal20}; \cite{Osetal20b}; \cite{Oseral21}).
\section{Interventions}
For an intervention to be considered evidence-based, it must be able to show
that they can effectively reduce a measure of symptoms for the target disorder
in a controlled trial (\cite{cook17}). Most interventions that are studied in
this way, and therefore fulfil this condition, are CBT-based and manualized
interventions (\cite{cook17}; \cite{shed18}). Because manualized therapies
tend to require a lower threshold of training in order to be implemented, they
are particularly suited to fulfil the needs of African governments and health
pracitioners in bridging the treatment gap (\cite{cook17}; \cite{douketal21};
\cite{lunetal14}; \cite{Osetal20}).\\
Generally, most of the data on low-cost interventions for low and middle income
countries comes from outside Africa, and while studies have been done more
recently in Africa, they are almost all from sub-Saharan Africa
(\cite{logetal18}). Most of the readily available studies discuss interventions
based on some form of cognitive behavioural therapy (CBT), are deployed by
lay health workers, and are low-intensity and rather short interventions
(\cite{chibandaetal11}; \cite{fernaetal21}; \cite{logetal18}; \cite{lunetal14}).
Out of all the articles reviewed, only Lofgren et al. (\citeyear{logetal18})
presented data on interventions other than psychotherapies.\\
Something important to note, most of the interventions discussed were not
targetting depression alone. Rather, some interventions targetted depression
alongside other common mental disorders (\cite{abasetal16}; \cite{chibandaetal11};
\cite{chibandaetal15}; \cite{chibandaetal16}; \cite{douketal21}), depression
specifically tied to living with HIV/AIDS (\cite{logetal18};
\cite{petersenetal14}), or otherwise targetted peri- and post-natal depression
(\cite{lunetal14}; \cite{nyatetal16}). This means that the studies might not be
as generalizable to other contexts, because the effect of the interventions are
not isolated to depression. However, real-life people live complex lives. As
established above, the prevalence of depression is high in African countries.
Furhermore, the link between depression and the other facets of peoples' lives
is intricately linked. The bi-directional relationship between povery and
mental illness is very well supported (\cite{lundetal10}; \cite{lund12};
\cite{ridetal20}; \cite{wahl17}).
\section{Quality of Evidence}
\section{Reflection}
\newpage
\printbibliography

\end{document}
