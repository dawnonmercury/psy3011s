\documentclass[stu,a4paper,12pt,donotrepeattitle]{apa7}
\usepackage{csquotes}
\usepackage[english]{babel}
\usepackage[
backend=biber,
style=apa,
]{biblatex}
\addbibresource{/home/dawn/.bibliography/psy3011s.bib}
\addbibresource{/home/dawn/.bibliography/psy2014s.bib}

\title{Essay Assignment}
\author{Dawn Opert}
\affiliation{OPRNET001}
\course{PSY3011S: Clinical Psychology II}
\professor{Katya Kee-Tui}
\duedate{2 September 2022}

\begin{document}
\maketitle
\section{Introduction}
Depression is a serious concern in Africa, and it is necessary to have a clear
idea about how countries across the continent are engaging with it, and what
Interventions can be effectively deployed to minimise it's impact on the mental
health burden. This essay will first discuss the prevalence of depression in
African countries. Next, this essay will discuss what Interventions have been
tested, whether they have been effective, and where they have been effective.
The quality of the evidence for these Interventions will be analysed, before
a final reflection will be made on the state of depression Interventions and
possible future directions for research and policy makers.\\
Depression and anxiety are among the most common mental disorders globally, and
cause a disproportionate amount of the global disability based healthcare
burden, a major contributor to morbidity, and a contributor to lower standard
of life (\cite{chibandaetal15}; \cite{chibandaetal16}; \cite{douketal21};
\cite{fernaetal21}; \cite{logetal18}; \cite{lunetal14}).\\
Africa bears the brunt of the burden of common mental disorders, with
countries across sub-Saharan Africa alone accounting for at least 19\% of the
global mental health burden (\cite{lunderal15}). In specific countries, at
least 16.5\% of adults in South Africa present with a common disorder
(\cite{lunetal14}), and at least 30\% of people attending primary healthcare
facilities in Zimbabwe present with depression and anxiety symptoms
(\cite{chibandaetal11}; \cite{chibandaetal15}).\\
Further, Africa has the highest rates of people living with HIV/AIDS, a
population that is especially vulnerable to depression as compared to a general
population, and the highest rate of people living with HIV related depression
(\cite{logetal18}; \cite{petersenetal14}). This is a serious problem because
depression is strongly associated with lower rates of antiretroviral therapy
adherence which is necessary for good management of HIV/AIDS (\cite{logetal18};
\cite{lunetal14}; \cite{petersenetal14}). On a similar front, depression
severely affects young and expecting mothers, increasing the risk of
complications during pregnancy, impairing caregivers' ability to create an
environment suitable for a child, and limiting caregivers' ability to cope with
parenthood (\cite{fernaetal21}; \cite{nyatetal16}).\\
Despite the high prevalence of depression and other common mental disorders
across Africa, there is a serious lack of access to treatment and support
(\cite{chibandaetal11}; \cite{chibandaetal15}; \cite{chibandaetal16};
\cite{douketal21}; \cite{fernaetal21}; \cite{logetal18}). This is exacerbated
by the fact that only half of African countries have a mental health policy to
speak of (\cite{lunetal14}; \cite{lunderal15}), and most governments do not have
the ability to allocate the necessary funds to implement one if they had it
(\cite{douketal21}; \cite{logetal18}). The general lack of resources and
infrastructure for providing effective treatments means that there is an urgent
need to develop interventions that work, that have lasting effects, and that can
be deployed cheaply and efficiently (\cite{douketal21}; \cite{lunetal14};
\cite{lunderal15}).
\section{Interventions}
For an intervention to be considered evidence-based, it must be able to show
that they can effectively reduce a measure of symptoms for the target disorder
in a controlled trial (\cite{cook17}). Most interventions that are studied in
this way, and therefore fulfil this condition, are based on Cognitive
Behavioural Therapy (CBT) and are manualized interventions
(\cite{cook17}; \cite{shed18}). Because manualized therapies tend to require a
lower threshold of training in order to be implemented, they are particularly
suited to fulfil the needs of African governments and health practitioners in
bridging the treatment gap (\cite{cook17}; \cite{douketal21}; \cite{lunetal14}).\\
Generally, most of the data on low-cost interventions for low and middle income
countries comes from outside Africa, and while studies have been done more
recently in Africa, they are almost all from sub-Saharan Africa
(\cite{logetal18}). As expected, most of the readily available data is on
CBT based interventions (\cite{chibandaetal11}; \cite{fernaetal21};
\cite{logetal18}; \cite{lunetal14}). Out of all the articles reviewed, only
Lofgren et al. (\citeyear{logetal18}) presented data on interventions other than
psychotherapies: six anti-depressant based interventions, a novel drug, an
exercise intervention, and three psychosocial interventions.\\
One of the key themes across these interventions was testing interventions that
could be delivered by minimally trained lay healthcare workers (LHWs) already
deployed in the field. Almost every single one of the psychotherapy interventions
followed this approach (\cite{abasetal16}; \cite{chibandaetal11};
\cite{chibandaetal15}; \cite{chibandaetal16}; \cite{douketal21};
\cite{fernaetal21}; \cite{logetal18}; \cite{lunetal14}; \cite{nyatetal16};
\cite{petersenetal14}). Lofgren et al. (\citeyear{logetal18}) also reported on
three anti-depressant interventions delivered by LHWs.\\
The reason given for this approach is that LHWs are already deployed across
primary healthcare clinics, and can therefore cheaply and efficiently deliver
interventions that act as a preventative measure for serious mental disorder
that would otherwise require (costly) hospitalization (\cite{abasetal16};
\cite{douketal21}; \cite{fernaetal21}; \cite{lunetal14}). Nyatsanza et al.
(\citeyear{nyatetal16}) also suggested that LHWs provide a built-in way to
ensure interventions are localized to and respect the cultural context where
they are being delivered.\\
Looking at some of the individual interventions, the \textit{Friendship Bench
Project} has by far the most data out of any of the reviewed interventions.
Three trials are reviewed here (\cite{chibandaetal11}; \cite{chibandaetal15};
\cite{fernaetal21}), the project is one of the oldest interventions tested in
an African context (\cite{abasal16}; \cite{chibandaetal11}; \cite{fernaetal21}).
\section{Quality of Evidence}
One of the most salient limitations of many interventions studies
is that the strict conditions of evidence gathering do not map onto their
practical, real world applications by their very nature (\cite{douketal21};
\cite{kaz14}; \cite{shed18}). However, none of the interventions discussed
targeted depression alone. Rather, some interventions targeted depression
alongside other common mental disorders (\cite{abasetal16}; \cite{chibandaetal11};
\cite{chibandaetal15}; \cite{chibandaetal16}; \cite{douketal21}); depression
specifically tied to living with HIV/AIDS (\cite{logetal18};
\cite{petersenetal14}); or otherwise targeted peri- and post-natal depression
(\cite{lunetal14}; \cite{nyatetal16}). This means that the studies might not be
as generalizable to other contexts, because the effect of the interventions are
not isolated to depression. However, real-life people live complex lives. As
established above, the prevalence of depression is high in African countries.
Furthermore, the link between depression and the other facets of peoples' lives
is intricately linked. The bi-directional relationship between poverty and
mental illness is very well supported (\cite{lundetal10}; \cite{lund12};
\cite{ridetal20}; \cite{wahl17}). People living with HIV/AIDS are also
more likely to develop depression (\cite{logetal18}).
\section{Reflection}
\newpage
\printbibliography

\end{document}
