\documentclass[stu,a4paper,12pt,donotrepeattitle]{apa7}
\usepackage{csquotes}
\usepackage[english]{babel}
\usepackage[
backend=biber,
style=apa,
]{biblatex}
\addbibresource{/home/dawn/.bibliography/psy3011s.bib}

\title{Tutorial Assignment 3}
\author{Dawn Opert}
\affiliation{OPRNET001}
\course{PSY3011S: Clinical Psychology II}
\professor{Katya Kee-Tui}
\duedate{12 October 2022}

\begin{document}
\maketitle
\section{Plagiarism Declaration}
\begin{enumerate}
    \item I know that plagiarism is wrong. Plagiarism means to present
        substantial portions or elements of another’s work, ideas or data
        as my own, even if the original author is cited occasionally.
    \item I have used the American Psychological Association formatting for
        citation and referencing. Each significant contribution to, and
        quotation in, this essay/project/report from the work or works of other
        people has been attributed, cited and referenced.
    \item This essay/project/report is my own work.
    \item \textbf{I have not allowed, and will not allow anyone to copy my work
        with the intention of passing it off as their own work.}
\end{enumerate}

\textbf{Signed:} \underline{Dawn Opert}\\
\textbf{Date:} \underline{12 October 2022}
\newpage
\section{Introduction}
Attenuated Psychosis Syndrome (APS) is a construct which attempts to describe
an individual who is in the earliest stages of a psychotic disorder, when
symptoms have begun to manifest but before they have reached the severity
threshold of a full disorder such as Schizophrenia (\cite{apa13};
\cite{corcetal21}; \cite{mahbell19b}; \cite{zachetal20}). Without delving too
deeply into the history, APS was developed in an effort to identify the early
signs of psychosis development, and to provide treatment in order to mitigate
the harm it causes and inhibit its development (\cite{carp20}; \cite{corcetal21};
\cite{mahbell19b}; \cite{zachetal20}). Originally, this syndrome was added to an
index of the DSM-5 which signalled the need for more research. Recently, there
have been calls for APS to be moved from this index into the main section of
the DSM-5 in the next DSM revision, giving rise to heated debate regarding how
wise this decision is (\cite{carp20}; \cite{corcetal21}; \cite{mahbell19b}). As
the background has been established, this paper will now outline the argument
for the inclusion of APS, the criticism and responses, and then conclude with
a possible remedy for the challenges APS faces.
\section{Support and Criticism for APS}
There are a number of immediate benefits to moving APS into the main section of
the DSM-5. In the case of a normal psychosis diagnosis, there has usually been
an enormous amount of harm to the client by the time the first psychotic
episode occurs (\cite{carp20}; \cite{zachetal20}). The primary benefit of APS
is that it facilitates the detection and treatment of developing psychotic
symptoms before they get to this point, saving the client from serious
complications and improving the chances for better results (\cite{carp20};
\cite{zachetal20}). Additionally, APS in the main DSM-5 will be a signal for
more focused research and clinician education on the period before a full
psychotic disorder is clearly recognisable, and a move away from framing
psychosis as all-or-nothing and discrete (\cite{carp20}; \cite{corcetal21};
\cite{zachetal20}).\\
However, there have been a number of issues raised with regards to APS.
Firstly, the criteria for APS are ambiguous, only requiring the presence of a
single symptom for diagnosis and relying on a vaguely defined threshold of a
client's insight into their symptom's non-reality (\cite{mahbell19b}).
Essentially, there is no way to clearly define the edges of APS, making its
use practically difficult (\cite{mahbell19b}). This compounds a second issue:
ambiguity makes misdiagnosis and false-positives more likely, leading to
potential sources of stigma (\cite{corcetal21}; \cite{mahbell19b};
\cite{zachetal20}). Further some clinicians may treat a diagnosis of APS as a
\textit{de facto} diagnosis of Schizophrenia, and treat it as such with
inappropriate prescription of antipsychotic medication, even in cases where
clinical guidelines refrain from their use (\cite{carp20}; \cite{corcetal21};
\cite{mahbell19b}; \cite{zachetal20}). Finally, there have been issues raised
regarding the reliability of APS as a predictor of Schizophrenia or other
psychotic disorder. Specifically, only a fraction of clients who could be
diagnosed with APS (around a quarter) go on to develop Schizophrenia, and of this
fraction there is a large number who functionally qualify for a full psychotic
disorder diagnosis already that is just being held at the levels of attenuated
psychotic symptoms by use of antipsychotic medication already (\cite{mahbell19b};
\cite{rabetal20}).\\
Inclusion in the main section of the DSM-5 will open the door to the
development of proper training and treatment guidelines (\cite{carp20};
\cite{corcetal21}; \cite{zachetal20}). This will in turn reduce the liklihood
of misdiagnosis as clinicians are properly trained, and will allow treatment
guidelines to solidify and prevent the inappropriate prescription of
antipsychotic medication (\cite{carp20}; \cite{corcetal21}; \cite{zachetal20}).
Furthermore, stigma seems to be more associated with the symptoms of psychotic
disorder than the label of psychotic disorders, so the risk regarding an APS
diagnosis should still be lower than with a full psychotic disorder, and in all
cases should be managed with education (\cite{carp20}; \cite{corcetal21}).
Finally, even a 10\% of individuals with APS progressing to a full psychotic
disorder is still significantly higher than the general population, and further
those that do not progress may either be worth studying for what they can tell us
about resiliency or otherwise may have been helped by treatment for APS in the
first place (\cite{carp20}; \cite{corcetal21}). Otherwise, research does
suggest that there is a high enough reliability and validity rate to begin
locking down a diagnosis definition for APS (\cite{rabetal20}).
\section{Conclusion}
While there is solid criticism of the move to add APS to the main section of
DSM-5, reasonable responses have been offered and recent research has closed
many of the gaps that previously existed. As discussed above, there are still
challenges that proponents of APS face, but these are not insurmountable. While
there are always issues associated with adding a new diagnosis to the DSM,
the inclusion of APS seems to be a net positive for the reasons outlined above.
Doing so will also put it in a position where the final gaps in the criteria
can be closed and research can be converted into practical clinical guidelines.
\newpage
\printbibliography
\end{document}
