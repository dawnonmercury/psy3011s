\documentclass[stu,a4paper,12pt]{apa7}
\usepackage{csquotes}
\usepackage[english]{babel}
\usepackage[
backend=biber,
style=apa]{biblatex}
\addbibresource{~/.bibliography/psy3011s.bib}

\title{Tutorial Assignment 1}
\author{Dawn Opert}
\affiliation{OPRNET001}
\course{PSY3011S: Clinical Psychology II}
\professor{Debbie Kaminer}
\duedate{25 August 2021}

\begin{document}
\maketitle
Good mental health is foundational to staying healthy, maintaining a state of
emotional security and wellbeing, and to the socio-economic options that are
available to a person across their life (\cite{barryetal13}). With that being
said, the state of mental health and access to mental health services in Lower-
and Middle-Income Countries (LMICs) is dire. Children and adolescents make up
a large part of the population in most LMICs, and depresssion and anxiety among
this demographic constitutes a sizable portion of the mental health burden
(\cite{barryetal13}; \cite{Oseral21}). However, access to the necessary care is
scarce in the communities that need it the most (\cite{Osetal20b};
\cite{Osetal20}; \cite{Oseral21}). It is necessary to find interventions that
are financially efficient and easily scalable within strained budgets, and
which are able to circumvent the social barriers to care (\cite{Osetal20b}).
Universal mental health promotion programmes in schools have the potential to
address some of the core problems outlined above, at the efficiency that is
necessary for effective deployment in a LMIC like South Africa. However, there
are still significant barriers to the implementation of these programmes. \\
Mental health promotion refers to interventions that focus on giving
participants the tools they need to live mentally healthy lives, and creating
an environment which naturally lends itself to positive mental health
(\cite{oretal18}). Because these programmes aim to prevent the development of
mental illness rather than treating problems after they have developed, they
can be employed universally, across an entire population rather than
only the people who are already at risk (\cite{Osetal20}; \cite{riduvetal11}). \\
In the countries that have the greatest share of the mental health burden,
there simply is not enough funding for mental care services. Structural issues
inhibit access to the services, and stigma surrounding mental health issues
prevents people from seeking care out of shame (\cite{barryetal13};
\cite{oretal18}; \cite{Osetal20b}). In South Africa specifically there is a
severe under-allocation of the health budget towards mental healthcare and an
overreliance on specialised facilities for treatment. This means an otherwise
progressive mental health policy is not actually implemented in areas that need
it most (\cite{docetal19}). Interventions need to be effective while also
remaining financially feasible for national budgets that are already under
immense strain. They must also effectively manage the stigma surrounding
recieving mental health care, to avoid alienating people who need that care
(\cite{Osetal20b}; \cite{Osetal20}; \cite{Oseral21}). \\
Another limitation that interventions employed in LMICs need to navigate is the
majority of measures in existing interventions have been overwhelmingly
developed for high income countries and Western contexts (\cite{barryetal13};
\cite{oretal18}; \cite{Osetal20}). \\
There are a number of universal mental health promotion programmes that attempt
to meet these challenges. By focusing on promoting good mental hygiene and
positive psychological concepts rather than relying on psychopathology, these
interventions can improve students' general mental health, and skills for
dealing with difficult emotional circumstances (\cite{barryetal13};
\cite{oretal18}; \cite{Osetal20}; \cite{Oseral21}; \cite{riduvetal11}).
Some interventions have been specifically tested in LMICs and multicultural
contexts, but there needs to be a continual effort to develope better suited
measures (\cite{barryetal13}; \cite{Osetal20}). \\
Two of the more promising programmes, the Shamiri group intervention
and the Resourceful Adolescent Program (RAP-A), were both delivered by
a limited number of trained lay-persons in a classroom context, mostly teachers
who did not require lengthy or expensive training to be able to effectively
implement them. They both included students who reported high depression and
anxiety symptoms and those who did not (\cite{Osetal20b}; \cite{Osetal20};
\cite{Oseral21}; \cite{riduvetal11}). \\
Shamiri was tested in Kenya, based on the idea of "wise interventions",
interventions that focus on singular concepts that contribute towards positive
mental health rather than teaching complicated systems all at once (\cite{Osetal20b};
\cite{Osetal20}; \cite{Oseral21}). The intervention taught three concepts over
a four week period: (1) Growth mindest, internal characteristics
such as skill and character traits are malleable rather than fixed; (2)
gratitude, teaching students the value of contemplating aspects of their lives
that they are grateful for; and (3) values or virtues, affirming the importance
of positive values and the protective power in cleaving to those values (\cite{Osetal20b}; \cite{Oseral21}).
This intervention had significant positive effects on students' general
wellbeing, and significant reductions in depresssion and anxiety symptoms
(\cite{Osetal20b}; \cite{Oseral21}). \\
RAP-A was implemented in a multicultural school in Mauritius, and focused on
teaching students strategies built on cognitive behavioural concepts to bolster
their self-esteem, problem solving skills, engaging with support networks, and
more over 11 one-hour sessions (\cite{barryetal13}; \cite{riduvetal11}). After a
follow-up, it was found that students who participated, whether originally
experiencing symptoms of depression or anxiey or not, reported much greater
resilience and general mental health (\cite{riduvetal11}). \\
Both of the programmes discussed were able to make a positive impact on the
students' lives, and were able to navigate the significant challenges they face.
Beccause they were generalized, both of the programmes circumvented the stigma
attached to their functions. Because they were implemented in school, they were
easily accessible to the students. Because they were facilitated by only a few
teachers who did not require intense training, they were financially and
logistically feasible (\cite{Osetal20b}; \cite{Oseral21}; \cite{riduvetal11}).
With that being said, more work must be done on refining these types of
programmes so that they are better able to withstand financial strain and
failures to correctly train teachers, and to further develop measures that
reflect many different South African (and other LMIC) contexts
(\cite{barryetal13}; \cite{oretal18}). Even so, classroom-based mental health
promotion programmes seem to be one of the most promising avenues for use in
South African school. These interventions have consistently shown that they can
be used at a low cost, with low logistical considerations, and with impressive
results. \\
South Africa and other LMICs face unique challenges to the implementation of
effective mental health care. Lack of funding, logistical challenges, social
issues and inequalities prevent mental health policies which are laudable on
paper from actualizing. Children and adolescents are especially vulnerable to
the gap in mental care. Universal mental health promotion, especially in
schools, are a potential answer to these challenges, as they focus on lowering
the potential incidence of mental illness among young people in the first place.
While these programmes need more refinement for the South African context, the
future for youth mental health could be bright.
   \newpage
   \printbibliography
\end{document}

%The school context contributes to this because (1) students are easily accessible
%for intervention during their lengthy school day, (2) teachers and school
%counsellors make for trusted figures for students, and (3) school systems
%potentially provide an infrastructure well suited to deliver interventions on a
%large scale.
